\chapter{Background}
\label{chp:ch2-background}

\section{Virtual Machines}
\label{sec:ch2-virtual-machines}

In this thesis we refer process virtual machines or application virtual machines simply as VMs.
They usually supports a single process running in a hosted operating system.
A VM provides a high-level abstraction composing a high-level programming language compared to the traditional low-level system programming languages.
This type of VM become popular since the wide adoption of the Java virtual machine which implements the Java programming language.
Microsoft's common language runtime is another example.
V8~\cite{v8}, a JavaScript implementation developed by Google, is a more recent popular language VM.

Virtual machines execute the hosted program in various fashions.
The VM can execute the hosted code using an interpreter, a just-in-time (JIT) compiler or a combination of both.
First the VM parses the targeted program from source code to a form of intermediate representation (IR).
The IR consists of series of ``instructions'' with each ``instruction'' representing exactly one functional operation, e.g., an arithmetic additional operation.
The interpreter then executes the program by translating the IR into actions one piece at a time.
For instance the interpreter interprets the IR instruction representing an additional operation by performing the actual addition.
The JIT compiler on the other hand execute the program by translating the IR into machine code and then redirect execution to the compiled machine code.

Interpreters often subject to suboptimal performance.
This is partially caused to the cost of by having to process each instruction before executing it.
Whereas a JIT compiler performs the translation prior to the execution of the program and avoid the overhead of processing repetitively executed instructions.
In addition, JIT compilers often implement more aggressive optimizations that potentially skip the execution of some portions of the program altogether.
However, the time the JIT compiler spend in compiling the program does not directly contribute to the execution of the target program.
This delay in compilation makes JIT compilers less ideal for programs that requires fast response.
A VM execution strategy called ``mixed mode'' addresses the disadvantages of both options by combining an interpreter and JIT compiler.
The VM initially executes the target program using the interpreter for fast response.
As the program becomes ``hot'', the VM switches the its execution by using the JIT compiler for fast execution.
The VM switches the execution strategy at the granularity of a compilation unit or a method as defined in the target language.
The initial use of interpreter also helps to collect more runtime information that can benefit optimizations later on performed by the JIT compiler.

\section{Interpreters}

\section{Just-In-Time Compilers}

\section{Type Specialization for Dynamic Languages}
