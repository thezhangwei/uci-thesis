\chapter{Introduction}
\label{chp:ch1-intro}

Since their first inception, dynamic languages or sometime referred as dynamically typed programming languages have enabled higher productivity for programmers.
They are no longer simply regard as ``scripting language'' used to accomplish relatively small tasks,
but have become ubiquitous in many domains including scientific computing and web programming.

GitHub~\cite{github} is a popular web-based open source software hosting service.
Among the six most poplar programming languages used in the projects hosted on GitHub, four of them are dynamic languages.
They are JavaScript, Python, PHP and Ruby.
PYPL~\cite{pypl} is a popular programming language index created by analyzing how often languages tutorials are searched on Google.
Among the top 16 languages ranked by PYPL, half of them are dynamic languages.
Many of the popular websites we are using today are built using dynamic languages.
For instance, the back-end of Airbnb and Hulu is written in Ruby and that of Quora and Reddit is written in Python.

Despite their popularity, performance has been the weakness of dynamic languages.
Languages like Python and Ruby are originally implemented as interpreters.
Although interpreters are easy to implement, their performance is suboptimal.
To address this weakness, we have seen recent works that have improved the performance of dynamic languages
by constructing a complete just-in-time (JIT) compilation based virtual machine for one particular language.
This approach offers promising performance benefit, but incurs significant implementation costs.

Alternatively, language implementors can build their languages on top of an existing mature virtual machine such as the Java Virtual machine (JVM).
In this way, the ``guest'' language can reuse the existing components of the ``hosting'' virtual machine to alleviate its implementation costs.
It also provide the opportunity for the ``hosted'' language to take advantage of the underlying JIT compiler to address its performance issue.
We explore the performance potential of ``hosted'' interpreters for dynamic languages.
We do so by hosting a ``highly dynamic'' language (Python) on the JVM, a VM for ``moderately dynamic'' languages.

This thesis makes the following contributions:
\begin{itemize}

\item A technique that speedups the execution of ``hosted'' bytecode interpreters using direct threading (Chapter~\ref{chp:ch3-bytecode}).

\item The first and fast Python 3 prototype implementation targeting the JVM (Chapter~\ref{chp:ch4-zippy}).

\item A new iterator optimization in the context of an optimizing Abstract Syntax Tree (AST) interpreter (Chapter~\ref{chp:ch5-peeling}).

\item A space efficient object model optimization for dynamic languages hosted on the JVM (Chapter~\ref{chp:ch6-object}).

\end{itemize}