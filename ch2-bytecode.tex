\chapter{Hosted Bytecode Interpreter}

A programming language interpreter executes programs in two steps.
First it parses the human readable source code, verifies its correctness and translates the code into a more efficient intermediate representation (IR) format.
The interpreter then picks up the translated program and executes it piece by piece.

Bytecode interpreters parse source program into bytecode, a highly compressed representation of the program.
The format of the bytecode is a form of virtual instruction set designed for this particular interpreter.
In the second step bytecode interpreters execute the bytecode as a sequence of virtual instruction one instruction at a time before finishing the last one.
Interpreters are also regard as virtual machines, since they emulate ``machines'' with their own virtual instruction sets.

In this Chapter, we go over the performance overheads of bytecode interpreters and the classic techniques used to overcome these overheads.
Lastly, we introduce ModularVM~\cite{savrun2013, savrun2014}, a research JVM that automatically optimize the performance of hosted bytecode interpreters.

\section{Performance Anatomy of Bytecode Interpreters}
