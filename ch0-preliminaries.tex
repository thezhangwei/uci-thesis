\thesistitle{Efficient Hosted Interpreter for Dynamic Languages}

\degreename{Doctor of Philosophy}

% Use the wording given in the official list of degrees awarded by UCI:
% http://www.rgs.uci.edu/grad/academic/degrees_offered.htm
\degreefield{Computer Engineering}

% Your name as it appears on official UCI records.
\authorname{Wei Zhang}

% Use the full name of each committee member.
\committeechair{Professor Michael Franz}
\othercommitteemembers
{
  Professor Kwei-Jay Lin\\
  Professor Guoqing Xu
}

\degreeyear{2015}

\copyrightdeclaration
{
  {\copyright} {\Degreeyear} \Authorname
}

% If you have previously published parts of your manuscript, you must list the
% copyright holders; see Section 3.2 of the UCI Thesis and Dissertation Manual.
% Otherwise, this section may be omitted.
% \prepublishedcopyrightdeclaration
% {
% 	Chapter 4 {\copyright} 2003 Springer-Verlag \\
% 	Portion of Chapter 5 {\copyright} 1999 John Wiley \& Sons, Inc. \\
% 	All other materials {\copyright} {\Degreeyear} \Authorname
% }

% The dedication page is optional.
% \dedications
% {
%   (Optional dedication page)
%
%   To ...
% }

\acknowledgments
{
  I would like to thank...

  (You must acknowledge grants and other funding assistance.

  You may also acknowledge the contributions of professors and
  friends.

  You also need to acknowledge any publishers of your previous
  work who have given you permission to incorporate that work
  into your dissertation. See Section 3.2 of the UCI Thesis and
  Dissertation Manual.)
}


% Some custom commands for your list of publications and software.
\newcommand{\mypubentry}[3]{
  \begin{tabular*}{1\textwidth}{@{\extracolsep{\fill}}p{4.5in}r}
    \textbf{#1} & \textbf{#2} \\
    \multicolumn{2}{@{\extracolsep{\fill}}p{.95\textwidth}}{#3}\vspace{6pt} \\
  \end{tabular*}
}
\newcommand{\mysoftentry}[3]{
  \begin{tabular*}{1\textwidth}{@{\extracolsep{\fill}}lr}
    \textbf{#1} & \url{#2} \\
    \multicolumn{2}{@{\extracolsep{\fill}}p{.95\textwidth}}
    {\emph{#3}}\vspace{-6pt} \\
  \end{tabular*}
}

% Include, at minimum, a listing of your degrees and educational
% achievements with dates and the school where the degrees were
% earned. This should include the degree currently being
% attained. Other than that it's mostly up to you what to include here
% and how to format it, below is just an example.
\curriculumvitae
{

\textbf{EDUCATION}

  \begin{tabular*}{1\textwidth}{@{\extracolsep{\fill}}lr}
    \textbf{Doctor of Philosophy in Computer Engineering} & \textbf{2015} \\
    \vspace{6pt}
    University of California, Irvine & \emph{Irvine, California} \\

    \textbf{Master of Science in Computer Engineering} & \textbf{2010} \\
    \vspace{6pt}
    Chalmers University of Technology & \emph{Gothenburg, Sweden} \\

    \textbf{Bachelor of Science in Mechanical Engineering} & \textbf{2004} \\
    \vspace{6pt}
    University of Science and Technology Beijing & \emph{Beijing, China} \\
  \end{tabular*}

\vspace{12pt}
\textbf{RESEARCH EXPERIENCE}

  \begin{tabular*}{1\textwidth}{@{\extracolsep{\fill}}lr}
    \textbf{Graduate Student Researcher} & \textbf{2010--2015} \\
    \vspace{6pt}
    University of California, Irvine & \emph{Irvine, California} \\

    \textbf{Master Student Researcher} & \textbf{2010} \\
    \vspace{6pt}
    Chalmers University of Technology & \emph{Gothenburg, Sweden} \\
  \end{tabular*}

\vspace{12pt}
\textbf{PROFESSIONAL EXPERIENCE}

  \begin{tabular*}{1\textwidth}{@{\extracolsep{\fill}}lr}
    \textbf{Software Development Intern} & \textbf{Summer 2014} \\
    \vspace{6pt}
    Oracle Labs & \emph{Belmont, CA} \\

    \textbf{Software Development Intern} & \textbf{Summer 2013} \\
    \vspace{6pt}
    Oracle Labs & \emph{Belmont, CA} \\

    \textbf{Customer Service Engineer} & \textbf{2007--2008} \\
    \vspace{6pt}
    ASML & \emph{Shanghai, China} \\

    \textbf{Production Engineer} & \textbf{2005--2007} \\
    \vspace{6pt}
    AT\&S & \emph{Shanghai, China} \\

    \textbf{Mechanical Design Engineer} & \textbf{2004--2005} \\
    \vspace{6pt}
    BMEI & \emph{Beijing, China} \\
  \end{tabular*}

\pagebreak

\textbf{PUBLICATIONS}

  % \mypubentry{Awesome paper}{Jun 2011}{Conference name}
  % \mypubentry{Another awesome paper}{Aug 2012}{Conference name}

Wei Zhang, Per Larsen, Stefan Brunthaler, Michael Franz.
\textbf{Accelerating Iterators in Optimizing AST Interpreters}.
\textit{In Proceedings of the 29th ACM SIGPLAN Conference on Object Oriented Programming:
Systems, Languages, and Applications, Portland, OR, USA, October 20-24, 2014 (OOPSLA '14)}, 2014.

G{\"u}lfem Savrun-Yeniçeri, Wei Zhang, Huahan Zhang, Eric Seckler, Chen Li, Stefan Brunthaler,
Per Larsen, Michael Franz.
\textbf{Efficient Hosted Interpreters on the JVM}.
\textit{In ACM Transactions on Architecture and Code Optimization, volume 11(1) pages 9:1–9:24}, 2014.

G{\"u}lfem Savrun-Yeniçeri, Wei Zhang, Huahan Zhang, Chen Li, Stefan Brunthaler, Per Larsen, Michael Franz.
\textbf{Efficient Interpreter Optimizations for the JVM}.
\textit{In Proceedings of the 10th International Conference on Principles and Practice of Programming in Java,
Stuttgart, Germany, September 11-13, 2013 (PPPJ '13)}, 2013.

\vspace{12pt}
\textbf{SOFTWARE}

\mysoftentry{ZipPy}{http://bitbucket.org/ssllab/zippy/}
{A fast and lightweight Python 3 implementation built using the Truffle framework.
It leverages the underlying Java JIT compiler and compiles Python programs to highly optimized machine code at runtime.}

\mysoftentry{ModularVM}{https://bitbucket.org/thezhangwei/modularvm/}
{An extension to the Maxine VM (Java Virtual Machine) that enables deeper integrations with JVM languages
like Jython (Python), Rhino (JavaScript) or JRuby (Ruby). It automatically accelerates guest language interpreters written in Java.}

}

% The abstract should not be over 350 words, although that's
% supposedly somewhat of a soft constraint.
\thesisabstract
{

Motivated by high development costs, production compilers and virtual machines, often support more than one language.
This strategy is most effective when the language family is homogeneous.
Many languages are very amenable to static program analysis, however, dynamic languages are not.
Consequently, a single VM cannot deliver peak performance for both types of languages without adapting its optimization strategy accordingly.

Informally, we host a ``highly dynamic'' language (Python) on the Java Virtual Machine, a VM for ``moderately dynamic'' languages.
While we are not the first to do so, our approach diverges from current practice by representing Python programs as abstract syntax trees, ASTs, rather than bytecode.
Not only are ASTs the simplest and most natural programming language implementation,
they also lend themselves well to optimizations those are particularly beneficial to highly dynamic languages.
Compared to Jython, which compiles Python programs to Java bytecode, our Python prototype is faster and requires less implementation effort.

}


%%% Local Variables: ***
%%% mode: latex ***
%%% TeX-master: "thesis.tex" ***
%%% End: ***
